\section{获取HSM状态}

\subsection{HSM状态}
这个struct是为了显示当前Lustre files状态,描述了当前的HSM状态和进程中的行为。\\
其结构与成员描述如下:
\begin{lstlisting}[language={c++},numbers=left]
	struct hsm_user_state {
		/* hus_states和hus_archive_id表示当前HSM状态 */
		// hus状态
		__u32			hus_states;     
		// hus归档ID
		__u32			hus_archive_id; 
		/*  若当前存在 HSM 行为 
			则用如下的四个元素:
			hus_in_progress_state, hus_in_progress_action,
			hus_in_progress_location和hus_extended_info数组
			描述该行为
		*/
		// hus进程中状态
		__u32			hus_in_progress_state;  
		// hus进程中行为
		__u32			hus_in_progress_action; 
		// hus进程中位置
		struct hsm_extent	hus_in_progress_location;   
		// hus扩展信息
		char			hus_extended_info[];    
	};
\end{lstlisting}


\subsection{通过文件标识符获取hsm状态}
\begin{lstlisting}[language={c++},numbers=left]
    /*  此API的目的是返回HSM状态和HSM请求
    API调用方给出一个文件标识符fd
    和一个hsm_user_state结构的指针hus
    然后通过hus指针返回该HSM状态。*/
    int llapi_hsm_state_get_fd(int fd, struct hsm_user_state *hus);
\end{lstlisting}

若该API返回int值为0则表示操作成功,为负则报错,负值为出错码。

\begin{lstlisting}[language={c++},numbers=left]
    /* 此API中,主要是调用对I/O通道进行管理的ioctl函数 */
    rc = ioctl(fd, LL_IOC_HSM_STATE_GET, hus);
\end{lstlisting}

而传入该ioctl函数中的为文件标识符fd\\和CMD命令LL\_IOC\_HSM\_STATE\_GET,此CMD命令参数为hus
\begin{lstlisting}[language={c++},numbers=left]
    #define LL_IOC_HSM_STATE_GET		_IOR('f', 211, struct hsm_user_state)
    /*  如上的define语句可看出,
    LL_IOC_HSM_STATE_GET命令实际为IO read命令
    传入ioctl的参数hus的作用为传回一个HSM状态 */
\end{lstlisting}


\subsection{通过文件路径获取HSM状态}

\begin{lstlisting}[language={c++},numbers=left]
    /* 此API的目的是返回HSM状态和HSM请求
    API调用方给出一个文件路径path
    和一个hsm_user_state结构的指针hus
    然后通过hus指针返回该HSM状态。 */
    int llapi_hsm_state_get(const char *path, struct hsm_user_state *hus);
\end{lstlisting}

而此API的具体实现是:
\begin{lstlisting}[language={c++},numbers=left]
    // 先通过路径名获取fd文件标识符 
    fd = open(path, O_RDONLY | O_NONBLOCK);
    // 再调用上述的get_fd函数获取hsm状态
    rc = llapi_hsm_state_get_fd(fd, hus);
\end{lstlisting}
