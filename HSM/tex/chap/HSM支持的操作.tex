\section{HSM支持的操作}

每个HSM Agent均需要向cdt注册。注册agent时需要指定--archieve参数。Lustre HSM功能支持最多32个带有编号的agent ID,编号从1开始。另外支持一个特殊的agent,该agent的编号为0,表示该agent可以用于服务所有的hsm后端。每个ID对应的agent可以注册无穷多个,lustre coordinator会根据一定的策略从所注册的agent中选出一个最适合的,之后向其发送hsm request。 

HSM的命令由客户端发起,具体的命令有以下几种: 

\begin{lstlisting}[language={c++},numbers=left]
    hsm_archive [--archieve=ID] FILE1 [FILE2…] 
    // 复制filelist中的文件下沉到HSM存储中,如果不指定--archieve,则默认使用ID=0

    hsm_release FILE1 [FILE2…] 
    // 删除lustre中对应该文件的数据部分。执行之前确保数据已被archive到HSM中

    hsm_restore FILE1 [FILE2…] 
    // 从HSM中恢复文件列表中的数据到lustre中

    hsm_cancel FILE1 [FILE2…] 
    // 取消针对文件列表正在执行的HSM请求
    
    hsm_action FILE […] 
    // 显示针对文件列表中的文件正在执行的hsm请求

    hsm_state  FILE […] 
    // 显示文件列表中的HSM状态
\end{lstlisting}

当前支持的状态有以下几种:
\begin{table}[!htb]
    \centering
    \resizebox{\textwidth}{!}{%
    \begin{tabular}{|c|c|c|}
    \hline
    state       & explanation                                & annotation     \\ \hline
    Non-archive & 标识为该状态的文件将不会被archive                       &                \\ \hline
    Non-release & 标识为该状态的文件将不会被release                       &                \\ \hline
    Dirty       & 存储在lustre中的文件以被更改,需要执行archive操作以便更新HSM中的副本 &                \\ \hline
    Lost        & 该文件已被archive,但是存储的HSM中的副本已丢失               & 打该标签时需要以root权限 \\ \hline
    Exist       & 文件的数据拷贝在HSM中已经存在                           &                \\ \hline
    Rleased     & 文件的LOV信息以及LOV对象已从lustre中移除                 &                \\ \hline
    Archived    & 文件数据已整个从lustre中拷贝至HSM中                     &                \\ \hline
    \end{tabular}%
    }
    \caption{HSM支持的状态}
    \label{tab:my-table}
\end{table}



\begin{lstlisting}[language={c++},numbers=left]
    hsm_set  [FLAGS] FILE […] 
    // 将文件列表中的文件打上上述标签
    
    hsm_clear [FLAGS] FILE […] 
    // 移除文件列表中相应的标签

    hsm_import path [flags…] 
    // 在lustre中创建文件存根,路径为path,
    // 其他信息由flags指定。文件的hsm状态为released,
    // 即文件的内容存储在后端存储,
    // 仅当文件被打开或者用户主动调用restore时文件的内容才被加载

\end{lstlisting}
 